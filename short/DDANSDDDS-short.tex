\documentclass[aspectratio=169,10pt,xcolor=x11names,english,french]{beamer}
\usepackage{babel}
\usepackage[autolanguage]{numprint}
\usepackage{amsmath}
\usepackage{currfile}                  % nom fichier de script
\usepackage{changepage}                % page licence
\usepackage{tabularx}                  % page licence
\usepackage{relsize}                   % \smaller et al.
\usepackage{awesomebox}                % boites signalétiques
\usepackage{fancyvrb}                  % texte verbatim
\usepackage{framed}                    % env. leftbar
\usepackage{pict2e}                    % cycle travail Git
\usepackage[overlay,absolute]{textpos} % couvertures
\usepackage{metalogo}       
\usepackage{textpos}           % logo \XeLaTeX

\usepackage{fontawesome}

\usepackage{stackengine, tikz} % for stacking logo

% For table
\usepackage{booktabs} % To thicken table lines
\usepackage{multirow}
\usepackage{amsmath}

\usepackage[group-separator={,}]{siunitx}

  %for tikz figure
\usepackage{tikz}
\usetikzlibrary{decorations.pathreplacing}
\usetikzlibrary{fadings}
\usetikzlibrary{positioning} %for lstm
\usetikzlibrary{shapes, backgrounds}
\usetikzlibrary{arrows}
\usetikzlibrary{chains}
\tikzstyle{line} = [draw, -latex']
\usepackage{adjustbox}

\usepackage{pgfplots}
\usepackage{pgfplotstable}

\definecolor{color3}{rgb}{0,0.7,0.3}
\definecolor{color1}{rgb}{0,0.1,0.8}
\definecolor{color2}{rgb}{0.9,0.0,0}

\graphicspath{{img/}}


%% =============================
%%  Informations de publication
%% =============================
\title{Reproductibilité en apprentissage automatique}
\author{David Beauchemin}
\date{30 octobre 2020}

%% =======================
%%  Apparence du document
%% =======================

%% Thème Beamer général
\usetheme[titleformat=allcaps, numbering=none, block=transparent]{metropolis}

%% Modifications aux couleurs: fond blanc, titres en texte noir sur
%% fond blanc
\setbeamercolor{normal text}{bg=white}
\setbeamercolor{frametitle}{fg=normal text.fg, bg=}

%% Déplacer les titres vers le bas sous la décoration du gabarit CFDD
\makeatletter
\setlength{\metropolis@frametitle@padding}{4.9ex}
\renewcommand{\metropolis@frametitlestrut@start}{
	\rule{0pt}{6ex +%
		\totalheightof{%
			\ifcsdef{metropolis@frametitleformat}{\metropolis@frametitleformat X}{X}%
		}%
	}
}
\renewcommand{\metropolis@frametitlestrut@end}{}
\makeatother

%% Format de la page de titre de section;
\makeatletter
\setbeamertemplate{section page}{%
	\begin{minipage}{22em}
		\raggedright
		\usebeamerfont{section title}
		\let\hyperlink\@secondoftwo\textcolor[cmyk]{0.67, 0.66, 0, 0.71}{\insertsectionhead}\par
	\end{minipage}
	\par
	\vspace{\baselineskip}
}
\makeatother

%% Polices de caractères
\newfontfamily\Overpass{Overpass}
\setsansfont{Overpass}        % police principale
\newfontfamily\OverpassSemiBold{Overpass}
[
BoldFont = *-SemiBold
]
\newfontfamily\OverpassExtraLightBold{Overpass}
[
UprightFont = *-ExtraLight,
BoldFont = *-Light
]
\newfontfamily\OverpassLight{Overpass}
[
UprightFont = *-Light,
BoldFont = *-Regular
]

%% Couleurs additionnelles
\definecolor{comments}{cmyk}{0,1, 1, 0}   % rouge foncé
\definecolor{link}{cmyk}{0.67, 0.66, 0, 0.71}    % liens internes
\definecolor{url}{cmyk}{0,1, 1, 0}       % liens externes
\definecolor{codebg}{named}{LightYellow1} % fond code R
\definecolor{rouge}{rgb}{0.85,0,0.07} % bandeau rouge UL
\definecolor{or}{rgb}{1,0.8,0}        % bandeau or UL
\definecolor{couleurpolice}{cmyk}{0.67, 0.66, 0, 0.71}

\colorlet{alert}{mLightBrown} % alias pour couleur Metropolis
\colorlet{dark}{mDarkTeal}    % alias pour couleur Metropolis
\colorlet{code}{mLightGreen}  % alias pour couleur Metropolis
\colorlet{shadecolor}{codebg}


\setbeamercolor{frametitle}{fg=couleurpolice}
\setbeamercolor{section in head/foot}{fg=couleurpolice}
\setbeamercolor{normal text}{fg=couleurpolice}



%% Hyperliens
\hypersetup{%
	pdfauthor = {David Beauchemin},
	pdftitle = {Reproductibilité en apprentissage automatique},
	colorlinks = {true},
	linktocpage = {true},
	allcolors = {link},
	urlcolor = {url},
	pdfpagemode = {UseOutlines},
	pdfstartview = {Fit},
	bookmarksopen = {true},
	bookmarksnumbered = {true},
	bookmarksdepth = {subsection}}

%% Paramétrage de babel pour les guillemets
\frenchbsetup{og=«, fg=»}

%% =========================
%%  Nouveaux environnements
%% =========================

%% Environnement pour le code informatique; hybride
%% des environnements snugshade* et leftbar de framed.
\makeatletter
\newenvironment{Scode}{%
	\def\FrameCommand##1{\hskip\@totalleftmargin
		\vrule width 3pt\colorbox{codebg}{\hspace{5pt}##1}%
		% There is no \@totalrightmargin, so:
		\hskip-\linewidth \hskip-\@totalleftmargin \hskip\columnwidth}%
	\MakeFramed {\advance\hsize-\width
		\@totalleftmargin\z@ \linewidth\hsize
		\advance\labelsep\fboxsep
		\@setminipage}%
}{\par\unskip\@minipagefalse\endMakeFramed}
\makeatother

%% Environnement pour le contenu d'un fichier; alias de snugshade*.
\newenvironment{Sfile}{\begin{snugshade*}}{\end{snugshade*}}

%% =====================
%%  Nouvelles commandes
%% =====================

%% Lien externe
\newcommand{\link}[2]{\href{#1}{#2~{\smaller\faExternalLink*}}}

%% Simili commande \HUGE
\newcommand{\HUGE}{\fontsize{36}{36}\selectfont}

%% Raccourcis usuels vg
\newcommand{\code}[1]{\textcolor{code}{\texttt{#1}}}

%%% =======
%%%  Varia
%%% =======

%% Longueurs pour la composition des pages couvertures avant et
%% arrière
\newlength{\banderougewidth} \newlength{\banderougeheight}
\newlength{\bandeorwidth}    \newlength{\bandeorheight}
\newlength{\imageheight}     \newlength{\imagewidth}
\newlength{\logoheight}

\begin{document}
	
	%% Style de l'entête et du pied de page
	
	%% frontmatter
	\input{couverture-avant}
	
	\begin{frame}{Situation}
		When assessing a commercial risk, an insurer needs to gather various information about the risk. This long and complex process implies numerous questions. Thus an insurer is prompt to use an external source to help reduce the number of questions.
		\\\bigskip
		Thus, we want to detect duplicate of commercial risk in another data source using as \textbf{little} as possible information \cite{davidduplicate}.
		
		\uncover<2->{\begin{block}{Example}\vspace{1pt}
				David Beauchemin, the owner of "Beauchemin inc.", calls for insurance. Using minimal information, we want to retrieve as much as possible from an external source to ask him as less than the necessary number of questions.
		\end{block}}
	\end{frame}
	
	\begin{frame}{How do we detect duplicate?}
		\only<1>{\begin{center}
			\begin{tikzpicture}[internal/.style={draw, rectangle, minimum width=4cm, inner sep=4pt},
			every text node part/.style={align=center}]
			\node[internal, align=center, minimum height=1.15cm](node1) at (-5, -2.75) {Duplicate \\ Detection Model};
			
			\node (node2) at (-1, -2.75) [cylinder, shape border rotate=90, draw, minimum height=1.15cm, minimum width=2cm, aspect=0.2] {Database 2};
			\draw[->,>=stealth] (node2) edge (node1.east);
			
			\node (node3) at (-9, -2.75) [cylinder, shape border rotate=90, draw, minimum height=1.15cm,minimum width=2cm, aspect=0.2] {Database 1};
			\draw[->,>=stealth] (node3) edge (node1.west);
			
			\node[internal, align=center, rounded corners=1ex, minimum height=1.15cm](node4) at (-5, -4.5) {Laval University};
			\draw[->,>=stealth] (node1) edge (node4);
			\end{tikzpicture}
		\end{center}}
		\note{Moreover, we consider only the top 1 document as the potential candidate.}
	\end{frame}
	
	\begin{frame}[label=REQ]\frametitle{Databases} 
		\begin{itemize}
			\item 	\textbf{Registre des Entreprises du Québec} ($\sim$3.5 millions entries)
					\begin{itemize}
						\item Names
						\item Address
						\item Economic activities
						\item Administrative informations
					\end{itemize}
			\item 	\textbf{Private dataset} (\num{21444} enterprises)
			\begin{itemize}
				\item Name
				\item Address
				\item Economic activity
			\end{itemize}
		\end{itemize}
	\end{frame}
	
	\begin{frame}[label=intact-REQ]\frametitle{Private Dataset and the REQ} 
			\num{11649} commercial risk are in the province of Quebec, \num{1706} were annotated
				\begin{itemize}
					\item \num{1418} \textit{(commercial risk, REQ entity)}
					\item 288 \textit{(commercial risk, None)}.
				\end{itemize}
			We only use the name and the address\footnote{Addresses and names are normalize (not included in the presentation).}.
	\end{frame}
	
	\section{Similarity Between Two Entities}
	
	\subsection{Similarity Algorithm}
	\begin{frame}[label=sim]\frametitle{Similarity Algorithm}	
		\begin{center}
			\begin{tikzpicture}[internal/.style={draw, rectangle, minimum width=3cm, minimum height=1.15cm, inner sep=4pt},
			every text node part/.style={align=center}]
			\node[internal, align=center](node1) at (-5, -2.75) {Similarity Algorithm};
			
			\node[internal, align=center, rounded corners=1ex, minimum height=1.15cm](node2) at (-0.75, -2.75) {Montreal \\ University};
			\draw[->,>=stealth] (node2) edge (node1.east);
			
			\node[internal, align=center, rounded corners=1ex, minimum height=1.15cm](node3) at (-9.25, -2.75) {Laval \\ University};
			\draw[->,>=stealth] (node3) edge (node1.west);
			
			\node[align=center](node4) at (-5, -4.5) {0.55};
			
			\draw[->,>=stealth] (node1) edge (node4);
			\end{tikzpicture}
		\end{center}
	\note{Their similarity ranks the documents, and we select the best one as the duplicate candidate.}
	\end{frame}

	\begin{frame}\frametitle{Similarity Algorithm}
        Similarities algorithms are one that measures the resemblance between two string base on the distance between their tokens.
        
        \note{10 different algorithms were used. Easy to used, No training, Give good results}
	\end{frame}
	
	\begin{frame}\frametitle{Name Interesting Results}
		\begin{table}[]
			\begin{tabular}{cccc}
				\toprule
				& StoS   & Jaro-Winkler & Jaccard \\
				\midrule
				Recall (\%)& 44.15 & 64.95 & \textbf{65.87}    
				\\\bottomrule
			\end{tabular}
		\end{table}
	\end{frame}
	
	\begin{frame}\frametitle{Address Interesting Results}
		\begin{table}[]
		\begin{tabular}{cccc}
				\toprule
				& StoS   & Jaro-Winkler & CSS \\
				\midrule
				Recall (\%) & 13.19 &  49.79 & \textbf{50.36}    
				\\\bottomrule
			\end{tabular}
		\end{table}
	\note{StoS give poor results using the normalized address due to the unomalized standard of writing (e.g., "qc" VS "(quebec)" and order of components).
	
	Using the address components without considering the order of the components improved results (AC).
	}
	\end{frame}
	
	\begin{frame}\frametitle{Missing Addresses}
		Since $16\%$ ($270$) of the pair \textit{(commercial risk, REQ entity)} are missing a address, the results are under-evaluated. For example, the CSS accuracy without those pairs is at $62.20\%$ near $12\%$ higher. \\\bigskip
		
		Those missing addresses are due to the confidentiality policy of the REQ \cite{guidereq}.
	\end{frame}
	
	\begin{frame}\frametitle{Similarity Algorithm Summary}
		Using either name or address, the similarities algorithms allows us to find the matched entity around $50\%$ of the time. \\ The leading approach been Jaccard using the name at near $67\%$.
	\end{frame}
	
	\subsection{Machine Learning}

	\begin{frame}\frametitle{Machine Learning Similarity Algorithm}
		\begin{center}
			\begin{tikzpicture}[internal/.style={draw, rectangle, minimum width=3cm, minimum height=1.15cm, inner sep=4pt},
			every text node part/.style={align=center}]
			\node[internal, align=center](node1) at (-5, -2.75) {Name-Address \\ Information Vector \\ Generator};
			
			\node[internal, align=center, rounded corners=1ex, minimum height=1.15cm](node2) at (-0.75, -2.75) {Montreal \\ University, \\ 2900 blvd \\ Edouard-Montpetit, \\ Montreal, QC};
			\draw[->,>=stealth] (node2) edge (node1.east);
			
			\node[internal, align=center, rounded corners=1ex, minimum height=1.15cm](node3) at (-9.25, -2.75) {Laval \\ University, \\ 2325 Rue de \\ l'Université, \\ Quebec, QC};
			\draw[->,>=stealth] (node3) edge (node1.west);
			
			\node[internal, align=center](node5) at (-5, -5.25) {Machine Learning \\ Model};
			
			\draw[->,>=stealth] (node1) edge (node5);
			
			\node[align=center](node4) at (-5, -7) {0.55};
			
			\draw[->,>=stealth] (node5) edge (node4);
			\end{tikzpicture}
		\end{center}
	\note{Allow us to use the name and the address simultaneously
		Generalization capability
	}
	\end{frame}
	
	\begin{frame}\frametitle{Name-Address Information Vector Generator}
		We used the previous similarity algorithm to generate an information vector between two entities using the name and address.
		\begin{block}{Example of an information vector}
			\resizebox{\textwidth}{!}{%	
				\begin{tabular}{ccccccc}
					StoS     & Levenshtein   & Jaro-Winkler & LCSP   & Jaccard & Cosinus   & -    \\\midrule
					0.00 & 0.15     &  0.25       & 0.35 & 0.15 & 0.15 & -  \\
					\cmidrule(lr){1-7}
					StoS     & Levenshtein  & Jaro  &LCSP   & Jaccard & Cosinus   & CSS   \\\midrule
					0.00  & 0.16  & 0.55       & 0.15 & 0.45 & 0.37 & 0.48 
				\end{tabular}
			} % end of scope of "\resizebox"  directive
		\end{block}
	\end{frame}

	\begin{frame}\frametitle{Results}
			\centering
			\begin{tabular}{cccc|c}
				\toprule
				(\%) &\begin{tabular}[c]{@{}c@{}}Logistic \\ Regression\end{tabular}  & \begin{tabular}[c]{@{}c@{}}Random \\ Forest\end{tabular} & \begin{tabular}[c]{@{}c@{}}Multilayer \\ perceptron\end{tabular} & Jaccard \\ \midrule
				Recall    & 66,67                                                                                                               & 73,54                                                   & \textbf{79,73}                                                            & 72,51  \\
				\bottomrule
			\end{tabular}
			\note{
				The random forest and the multilayer perceptron achieve better recall than Jaccard. The best being the perceptron with near $80\%$ recall.
					}
	\end{frame}

	\begin{frame}\frametitle{Improving the Results - $N$ Most Similar}
		We consider a matching is good when the pair \textit{(commercial risk, REQ entity)} is included in the $N$ most similar.
	\end{frame}

	\begin{frame}\frametitle{Results}
			\begin{tikzpicture}
			\begin{axis}[grid style={dashed,gray!50}, axis y line*=left, axis x line*=bottom, ybar, nodes near coords=\rotatebox{90}{\pgfmathprintnumber\pgfplotspointmeta}, nodes near coords style={font=\tiny}, every axis plot/.append style={line width=1.25pt, mark size=0pt}, width=\textwidth, height=.55\textwidth, grid=none, xtick=data, symbolic x coords={LR, RF, MP, Jaccard}, ylabel=Recall (\%)]
			\addplot[color3, fill,  opacity=0.5] table[x=x6, y=y6, col sep=comma]{./data/plot_rappel_top_n.csv};
			\addplot[color1, fill,  opacity=0.5] table[x=x7, y=y7, col sep=comma]{./data/plot_rappel_top_n.csv};
			\addplot[color2, fill,  opacity=0.5] table[x=x8, y=y8, col sep=comma]{./data/plot_rappel_top_n.csv};
			\addlegendentry{Top 1};
			\addlegendentry{Top 5};
			\addlegendentry{Top 10};
			\end{axis}
			\end{tikzpicture}
			
			\note{ Using a top $N$ approach greatly improved the results. 
				 Using $N = 10$, we can achieve a near max recall at $~91\%$ with the multilayer perceptron (max of $93\%$ with the indexing)}
	\end{frame}

	\begin{frame}\frametitle{Machine Learning Algorithm Summary}
		The used of name and address simultaneously with a machine learning algorithm improved the results. \\
		Using a top $ N $ approach helps achieve better results when $ N $ is greater than 1.
	\end{frame}

	\begin{frame}\frametitle{Future Works}
		\begin{itemize}
			\item Word embeddings \cite{word2vec, glove, wu2017starspace, name2vec, singh2019embedding}
			\item Siamese Network \cite{godbole2018siamese, 8967103}
			\item Uses of spatial data \cite{10.1145/1183471.1183486}
			\item Removal of more specitif stop words using a TF-IDF approach \cite{tfidf}
		\end{itemize}
	\end{frame}

	\begin{frame}\frametitle{Conclusion}
		\begin{itemize}
			\item We have shown that using a similarity algorithm can achieve good results.
			\item Uses of machine learning algorithm (such as multilayer perceptron) can achieve greater results.
			\item Using a $N$ most similar approach, where $N$ is greater than one, help improved the results, achieving almost the max recall value.
		\end{itemize}
	\end{frame}

	\begin{frame}{Acknowledgments}

		This research was supported by the Natural Sciences and Engineering Research Council of Canada (IRCPJ 529529-17) and a Canadian insurance company. 
		\\
		
		Luc Lamontagne for his mentorship and reviewers for their comments.
	\end{frame}
	
	\input{couverture-arriere}

	\begin{frame}[t, allowframebreaks]
		\frametitle{References}
		\bibliographystyle{apalike}
		\bibliography{DDANSDDDS}
	\end{frame}
	
\end{document}